% Definiciones y constantes de estilo
% Clase del documento
\documentclass[a4paper,12pt,twoside,openright,titlepage]{book}

% Paquetes necesarios
\usepackage{eurosym}
\usepackage[utf8]{inputenc}
\usepackage[spanish]{babel}
\usepackage[hidelinks,colorlinks,citecolor=Fuchsia,urlcolor=blue,linkcolor=Cerulean]{hyperref}
\usepackage{listings}
\usepackage{color}
\usepackage{anysize}
\usepackage{fancyhdr}
\usepackage{cite}
\usepackage{multirow}
\usepackage{titlesec}
\usepackage[cmex10]{amsmath}
\usepackage{algorithmic}
\usepackage{textcomp}
\usepackage{enumerate}
\usepackage{fixltx2e}
\usepackage{emptypage}
\usepackage{float}
\usepackage[pdftex]{graphicx}
\usepackage{array}
\usepackage{mdwmath}
\usepackage[caption=false,font=footnotesize]{subfig}
\usepackage{fixltx2e}
\usepackage{pdfpages}
\usepackage{quotchap}
\usepackage{fancybox}
\usepackage[acronym]{glossaries}
\usepackage{appendix}
\usepackage{bookmark}

% Euro (€)
\DeclareUnicodeCharacter{20AC}{\euro}

% Estilo de la bibliografía
\bibliographystyle{IEEEtran}

% Inclusión de gráficos
\graphicspath{{./graphics/}}

% Extensiones de gráficos
\DeclareGraphicsExtensions{.pdf,.jpeg,.jpg,.png}

% Definiciones de colores (para hidelinks)
\definecolor{LightCyan}{rgb}{0,0,0}
\definecolor{Cerulean}{rgb}{0,0,0}
\definecolor{Fuchsia}{rgb}{0,0,0}

% Keywords (español e inglés)
\def\keywordsEn{\vspace{.5em}
{\textbf{\textit{Key words ---}}\,\relax%
}}
\def\endkeywordsEn{\par}

\def\keywordsEs{\vspace{.5em}
{\textbf{\textit{Palabras Clave ---}}\,\relax%
}}
\def\endkeywordsEs{\par}


% Abstract (español e inglés)
\def\abstractEs{\vspace{.5em}
{\textbf{\textit{Resumen ---}}\,\relax%
}}
\def\endabstractEs{\par}

\def\abstractEn{\vspace{.5em}
{\textbf{\textit{Abstract ---}}\,\relax%
}}
\def\endabstractEn{\par}

% Estilo páginas de capítulos
\fancypagestyle{plain}{
\fancyhf{}
\fancyfoot[CO]{\footnotesize\emph{\nombretrabajo}}
\fancyfoot[RO]{\thepage}
\renewcommand{\footrulewidth}{.6pt}
\renewcommand{\headrulewidth}{0.0pt}
}

% Estilo resto de páginas
\pagestyle{fancy}

% Estilo páginas impares
\fancyfoot[CO]{\footnotesize\emph{\nombretrabajo}}
\fancyfoot[RO]{\thepage}
\rhead[]{\leftmark}

% Estilo páginas pares
\fancyfoot[CE]{\emph{\pieparcen}}
\fancyfoot[LE]{\thepage}
\fancyfoot[RE]{\pieparizq}
\lhead[\leftmark]{}

% Guía del pie de página
\renewcommand{\footrulewidth}{.6pt}

% Nombre de los bloques de código
\renewcommand{\lstlistingname}{Algoritmo}

% Definiciones de funciones para los títulos
\newlength\salto
\setlength{\salto}{3.5ex plus 1ex minus .2ex}

\newlength\resalto
\setlength{\resalto}{2.3ex plus.2ex}

% Corrección warning
\setlength{\headheight}{15pt} 

% Estilo de sección
\newcommand{\lsection}[1]
                {\section{#1}
                \vskip-.9\resalto % Corrección del posible salto por defecto de \section
                \hrule
                \vskip+.9\salto} % Vuelvo ha realizar el salto

% Estilo de los acrónimos
\renewcommand{\acronymname}{Glosario}
\renewcommand{\glossaryname}{Glosario}
\pretolerance=2000
\tolerance=3000

% Pie de tabla
\addto\captionsspanish{
\def\tablename{Tabla}
\def\listtablename{\'Indice de tablas}
}

% Traducir appendix/appendices
\renewcommand\appendixtocname{Apéndices}
\renewcommand\appendixpagename{Apéndices}

% Definiciones de comandos
\newcommand{\nombreautor}{TODO: Nombre Apellido1 Apellido2}
\newcommand{\nombretutor}{TODO: NombreTutor Apellido1 Apellido2}
\newcommand{\nombretrabajo}{TODO: Título del Trabajo de Fin de Grado}
\newcommand{\fecha}{TODO: Fecha de entrega}
\newcommand{\grado}{TODO: Grado}
\newcommand{\grupoInvestigacion}{TODO: Grupo de investigación}
\newcommand{\departamento}{TODO: Departamento}
\newcommand{\facultad}{Escuela Politécnica Superior}
\newcommand{\universidad}{Universidad Autónoma de Madrid}
\newcommand{\pieparizq}{TODO: Pie de página par}
\newcommand{\pieparcen}{Trabajo de Fin de Grado}
\newcommand{\logoizq}{Logo_EPS}
\newcommand{\logoder}{Logo_UAM}
\newcommand{\correo}{TODO: Correo de contacto}

% Glosario y acrónimos
\makeglossaries
% Acrónimos

% TODO: Añadir aquí los acrónimos
% Ejemplo de acrónimo
\newacronym{FPGA}{FPGA}{Field-Programable Gate Array}

% Glosario

% TODO: Añadir aquí las definiciones del glosario
% Ejemplo de glosario
\newglossaryentry{Bitstream}{name={Bitstream},description={En este contexto se refiere al binario que configura el Hardware de la FPGA}}

% Inicio del documento
\begin{document}

% Elección del idioma (español)
\selectlanguage{spanish}

%
% Portada
%
\pagenumbering{gobble}
%
% Portada
%

% Universidad, Facultad
\begin{titlepage}
\selectlanguage{spanish}
\begin{center}
\textbf{\begin{huge}
\universidad \\
\end{huge}}
\bigskip 
\begin{LARGE}
\facultad \\
\end{LARGE}
\end{center}

\bigskip
\bigskip

%
% Imágenes (logos) izquierdo y derecho
%
\begin{figure}[h]
	\begin{center}
		\includegraphics[scale=0.35]{\logoizq}
    \hspace{1cm}
		\includegraphics[scale=0.4]{\logoder}
	\end{center}	
\end{figure}

\bigskip
\bigskip
\bigskip

% Grado
\begin{center}
\begin{large}
\textbf{\grado}\\
\end{large}
\end{center}

\bigskip

\textbf{\begin{center}
\begin{huge}
\MakeUppercase{Trabajo de Fin de Grado}
\end{huge}
\end{center}}

\bigskip
\bigskip

% Nombre del TFG
\begin{center}
\textbf{\begin{large}
\MakeUppercase{\nombretrabajo}\\
\end{large}}
\end{center}

% Nombre del autor
\vspace{\fill}
\begin{center}
\textbf{\nombreautor}\\
% Tutor
\textbf{Tutor: \nombretutor}\\
% Ponente, si está definido en main.tex
\ifcsname nombreponente\endcsname
\textbf{Ponente: \nombreponente}\\
\fi

\bigskip

% Fecha
\textbf{\fecha}\\
\end{center}
\end{titlepage}

% Primera página
\pagenumbering{Alph}
\thispagestyle{empty}
\par\vspace*{\fill}
\begin{flushleft}
\begin{scriptsize}
\end{scriptsize}\end{flushleft}
\newpage
\thispagestyle{empty}
\begin{center}

% Nombre del trabajo
\textbf{\begin{large}
\MakeUppercase{\nombretrabajo}\\*
\end{large}}
\vspace*{0.2cm}
\vspace{5cm}

% Nombre del autor y del tutor
\large Autor: \nombreautor \\*
\large Tutor: \nombretutor \\*
\ifcsname nombreponente\endcsname
\large Ponente: \nombreponente\\
\fi

\vfill

% Grupo de investigación, departamento, facultad, universidad y fecha
\ifcsname grupoInvestigacion\endcsname
\grupoInvestigacion \\
\fi
\departamento \\
\facultad \\
\universidad \\
\vspace{1cm}
\fecha \\

\clearpage

\end{center}
\normalsize

\hypersetup{pageanchor=true}

%
% Agradecimientos
%
\pagenumbering{Roman}
\setcounter{page}{0}
\chapter*{Agradecimientos}

TODO: Agradecimientos.

Lorem ipsum dolor sit amet, consectetur adipiscing elit. Phasellus laoreet dolor at sodales porta. Morbi facilisis hendrerit lacus vel sollicitudin. Aenean eleifend urna metus, eget vestibulum libero dictum tincidunt. Curabitur quis ultrices lorem. Duis ultricies, eros eget condimentum pharetra, tellus eros lobortis nulla, vel mattis nibh dui et felis. Interdum et malesuada fames ac ante ipsum primis in faucibus. Nam non lorem et ligula condimentum molestie. Fusce quis dolor non metus suscipit commodo. Praesent vel pulvinar lectus. Nullam ac dui eget magna accumsan volutpat. Aliquam sed purus quis lorem dictum rutrum auctor eu enim. Pellentesque a urna ac ligula cursus lacinia. Aenean sodales justo massa, vel imperdiet justo imperdiet ut. Nulla euismod pulvinar arcu eu convallis. Vivamus a tempus nunc, et vulputate nulla.

Sed dapibus aliquam imperdiet. Vivamus est quam, fermentum vitae augue id, ultricies tincidunt massa. Praesent tincidunt ex sem, ut aliquet nulla imperdiet eu. Duis ac ultricies lorem. Aenean consequat ipsum nec arcu aliquam, sit amet interdum quam tempus. In justo odio, bibendum vel nulla nec, aliquet tristique justo. In vel metus ut libero suscipit ultricies.

Class aptent taciti sociosqu ad litora torquent per conubia nostra, per inceptos himenaeos. Proin urna elit, iaculis id quam at, pretium laoreet ipsum. Phasellus ultricies faucibus ex et eleifend. Quisque facilisis erat dolor, ac rhoncus erat convallis et. Aliquam semper eleifend imperdiet. Sed eros ipsum, sagittis in pellentesque vel, vestibulum a augue. Duis sapien mauris, fringilla a tortor ut, sollicitudin volutpat nunc. Pellentesque vestibulum vel arcu in molestie. Nullam fermentum dolor luctus metus efficitur pulvinar. Pellentesque risus enim, tempus id ullamcorper in, maximus id nisl. Cras rhoncus consequat augue eu gravida. Ut efficitur mauris vitae orci dignissim sagittis. Suspendisse vitae massa eget nunc bibendum interdum.

Vivamus congue tellus nec lobortis feugiat. Nam hendrerit ullamcorper tempus. Proin maximus, lacus at tempor pellentesque, sem nisi facilisis lorem, sagittis tristique mauris dui at est. Class aptent taciti sociosqu ad litora torquent per conubia nostra, per inceptos himenaeos. Mauris pellentesque lobortis leo, ac dictum urna tempus id. Curabitur sed ante leo. Proin laoreet nisi nec dictum auctor. Mauris lacinia erat ut massa viverra, nec tempus metus elementum. Cras ut blandit justo, in pretium massa. In hac habitasse platea dictumst. Donec malesuada viverra quam, in ultricies libero. Phasellus finibus velit in sem tempus mattis at tristique ligula.

% Cita
\begin{flushright}
\textit{``TODO: Cita relevante''}
TODO: Autor de la cita
\end{flushright}
  

%
% Resumen
%
% Resumen en inglés
\chapter*{Abstract}

\begin{abstractEn}
TODO: Resumen en inglés, 250-500 palabras.

Lorem ipsum dolor sit amet, consectetur adipiscing elit. Aliquam malesuada libero auctor sapien volutpat, sed fringilla enim tristique. Aliquam varius lorem in risus tempus egestas. Aenean accumsan elementum diam vel commodo. Nulla lectus sapien, finibus ac mauris non, efficitur venenatis felis. Donec at rutrum dolor, a lobortis arcu. In fermentum hendrerit bibendum. Phasellus eget arcu quam. Maecenas vulputate sapien eu dictum pulvinar. Suspendisse sit amet neque a turpis efficitur dapibus ut et turpis.

Vestibulum commodo faucibus tellus vitae consequat. Donec purus enim, hendrerit vitae feugiat sed, sagittis in tortor. Duis sed ex non ligula cursus dapibus. Etiam pellentesque suscipit dolor, vel facilisis est ornare sed. Nullam eleifend tellus non elementum efficitur. Donec semper felis ac porttitor ultricies. Vestibulum sodales justo nisl, in egestas lacus egestas nec. Fusce faucibus felis lacus, sit amet placerat justo porta vitae. Nullam volutpat viverra lorem quis euismod. Duis felis erat, dictum et sem vitae, fringilla ultrices dui. Morbi mattis arcu at orci accumsan facilisis. Aenean tortor velit, hendrerit id vulputate ac, sagittis nec libero. Donec elementum dolor orci, a mattis augue lobortis nec. Suspendisse vulputate, diam vel accumsan pellentesque, ex purus volutpat ipsum, vel luctus urna sem non turpis. Donec vitae molestie odio.

Donec lobortis, eros non sodales dapibus, ex eros sollicitudin tortor, ut vulputate massa nibh sit amet ipsum. Sed a lectus eu diam pretium vestibulum. Pellentesque finibus, felis ac finibus vulputate, libero mauris placerat nulla, ut vestibulum ante metus ut neque. Aliquam tempus tortor ac mauris pulvinar iaculis. Vivamus pretium id libero sed tempus. Donec tincidunt turpis tempor vehicula egestas. Vestibulum elementum, urna non tincidunt tempus, risus ipsum posuere felis, ac suscipit diam nunc et neque. Vestibulum faucibus leo vel nibh tempor tincidunt. Nullam nunc augue, aliquet in congue nec, gravida at risus. Proin semper iaculis nisi vitae imperdiet. Suspendisse sed risus feugiat, dapibus sapien quis, pulvinar turpis.

\end{abstractEn}

% Palabras clave en inglés
\begin{keywordsEn}
TODO: Palabras clave en inglés, separadas por coma.
\end{keywordsEn}

% Resumen en español
\chapter*{Resumen}

\begin{abstractEs}
TODO: Resumen en español, 250-500 palabras.

Lorem ipsum dolor sit amet, consectetur adipiscing elit. Aliquam malesuada libero auctor sapien volutpat, sed fringilla enim tristique. Aliquam varius lorem in risus tempus egestas. Aenean accumsan elementum diam vel commodo. Nulla lectus sapien, finibus ac mauris non, efficitur venenatis felis. Donec at rutrum dolor, a lobortis arcu. In fermentum hendrerit bibendum. Phasellus eget arcu quam. Maecenas vulputate sapien eu dictum pulvinar. Suspendisse sit amet neque a turpis efficitur dapibus ut et turpis.

Vestibulum commodo faucibus tellus vitae consequat. Donec purus enim, hendrerit vitae feugiat sed, sagittis in tortor. Duis sed ex non ligula cursus dapibus. Etiam pellentesque suscipit dolor, vel facilisis est ornare sed. Nullam eleifend tellus non elementum efficitur. Donec semper felis ac porttitor ultricies. Vestibulum sodales justo nisl, in egestas lacus egestas nec. Fusce faucibus felis lacus, sit amet placerat justo porta vitae. Nullam volutpat viverra lorem quis euismod. Duis felis erat, dictum et sem vitae, fringilla ultrices dui. Morbi mattis arcu at orci accumsan facilisis. Aenean tortor velit, hendrerit id vulputate ac, sagittis nec libero. Donec elementum dolor orci, a mattis augue lobortis nec. Suspendisse vulputate, diam vel accumsan pellentesque, ex purus volutpat ipsum, vel luctus urna sem non turpis. Donec vitae molestie odio.

Donec lobortis, eros non sodales dapibus, ex eros sollicitudin tortor, ut vulputate massa nibh sit amet ipsum. Sed a lectus eu diam pretium vestibulum. Pellentesque finibus, felis ac finibus vulputate, libero mauris placerat nulla, ut vestibulum ante metus ut neque. Aliquam tempus tortor ac mauris pulvinar iaculis. Vivamus pretium id libero sed tempus. Donec tincidunt turpis tempor vehicula egestas. Vestibulum elementum, urna non tincidunt tempus, risus ipsum posuere felis, ac suscipit diam nunc et neque. Vestibulum faucibus leo vel nibh tempor tincidunt. Nullam nunc augue, aliquet in congue nec, gravida at risus. Proin semper iaculis nisi vitae imperdiet. Suspendisse sed risus feugiat, dapibus sapien quis, pulvinar turpis.

\end{abstractEs}

% Palabras clave en español
\begin{keywordsEs}
TODO: Palabras clave en español, separadas por coma.
\end{keywordsEs}


%
% Glosario
%
\printglossary[title=Glosario,toctitle=Glosario]
\printglossary[title=Acrónimos,toctitle=Acrónimos,type=\acronymtype]

%
% Tabla de contenidos
%
\tableofcontents
\listoftables
\listoffigures
\cleardoublepage

% Estilo de párrafo de los capítulos
\setlength{\parskip}{0.75em}
\renewcommand{\baselinestretch}{1.25}
% Interlineado
\spacing{1.3}
% Numeración contenido
\pagenumbering{arabic}
\setcounter{page}{1}

%
% Introducción
%
\pagenumbering{arabic}
\setcounter{page}{1}
\chapter{Introducción}

TODO: Introducción del trabajo/proyecto

\section{Alcance}

TODO: Alcance del trabajo/proyecto

\section{Estructura del documento}

TODO: Descripción de la estructura del documento
%
% Estado del arte
%
\chapter{Estado del Arte\label{sec:estado_del_arte}}

TODO: Estado del arte


%
% Diseño
%
\chapter{Diseño\label{sec:disenho}}

TODO: Diseño del proyecto

%
% Desarrollo
%
\chapter{Desarrollo\label{sec:desarrollo}}

TODO: Desarrollo del proyecto

%
% Resultados
%
\chapter{Resultados\label{sec:resultados}}

TODO: Pruebas y resultados

%
% Conclusiones
%
\chapter{Conclusiones\label{sec:conclusiones}}

TODO: Conclusiones sobre el trabajo realizado 

%
% Página en blanco
%
\cleardoublepage

%
% Bibliografía
%
\bibliography{src/bibliografia}
\addcontentsline{toc}{chapter}{Bibliografía}

% No expandir elementos para llenar toda la página
\raggedbottom

%
% Apéndices
%
\appendix
\cleardoublepage
\addappheadtotoc
\appendixpage

%
% TODO: Apéndices del TFG
%
\chapter{Ejemplos de bloques y comandos útiles en LaTeX\label{sec:ejemplos}}
\section{Ejemplo de sección}

%
% Breve guía de comandos útiles para la memoria
%

% Citar una referencia
La DARPA creó el protocolo de Internet \cite{ipv4sta}.

% Citar un elemento del glosario
Citamos el acrónimo \gls{FPGA}.

% Citar un elemento del glosario (primera letra en may´usculas)
\Gls{bitstream} es una secuencia de bits.

% Insertar una imagen con pie de página
\begin{figure}[htp!]
  \centering
  \includegraphics[width=0.75\textwidth,clip=true]{Logo_UAM}
  \caption{Logo de la Universidad Autónoma de madrid.}
  \label{fig:logo_uam}
\end{figure} 

% Referenciar una etiqueta (label)
La figura~\ref{fig:logo_uam} se utiliza en la portada.

% Nueva página
\clearpage

% Añadir código fuente sin líneas
\begin{lstlisting}[label=algoritmo:quicksort,language=C,frame=single,caption=Algoritmo de ordenación Quicksort]
#include <stdio.h>
 
void quick_sort (int *a, int n) {
    int i, j, p, t;
    if (n < 2)
        return;
    p = a[n / 2];
    for (i = 0, j = n - 1;; i++, j--) {
        while (a[i] < p)
            i++;
        while (p < a[j])
            j--;
        if (i >= j)
            break;
        t = a[i];
        a[i] = a[j];
        a[j] = t;
    }
    quick_sort(a, i);
    quick_sort(a + i, n - i);
}
\end{lstlisting}

% Bloque de código inseparable
\begin{code}
#include <stdio.h>
 
void quick_sort (int *a, int n) {
    int i, j, p, t;
    if (n < 2)
        return;
    p = a[n / 2];
    for (i = 0, j = n - 1;; i++, j--) {
        while (a[i] < p)
            i++;
        while (p < a[j])
            j--;
        if (i >= j)
            break;
        t = a[i];
        a[i] = a[j];
        a[j] = t;
    }
    quick_sort(a, i);
    quick_sort(a + i, n - i);
}
\end{code}

% Fórmula dentro de una línea de texto
La ecuación de Euler ($e^{ \pm i\theta } = \cos \theta \pm i\sin \theta$) es citada frecuentemente como un ejemplo de belleza matemática.

% Fórmula independiente
\begin{equation}\label{eq:pythagoras}
a^2 + b^2 = c^2
\end{equation}


% Fin del documento
\end{document}
